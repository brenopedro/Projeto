\chapter*{Preface}
\addcontentsline{toc}{chapter}{Preface}

\ \ \ \ Era uma noite chuvosa, com relâmpagos e trovões. Tarka estava sentado com as pernas cruzadas e olhos fechados à beira de um desfiladeiro com um escudo encravado no chão a sua frente. A sua aparência não era mais velha que um homem de trinta anos, musculoso, tinha um cabelo longo e uma barba espessa como se não os aparasse em anos, ambos brancos. À sua frente, tinha um desfiladeiro que não era possível ver o fundo, talvez até mesmo não tivesse, mas tinha apenas três metros de largura, e do outro lado haviam criaturas esperando que o homem saísse de lá ou morresse para que pudessem passar. Tais monstros não teriam problemas em pular o vazio que se estendia até o fundo, na verdade não era nada para eles, entretanto havia uma barreira que rodeava tudo, não os deixando passar e eles sabiam que não era possível quebrá-la tão fácil, pois muitos haviam tentado e caído no abismo sem fim.

De repente, os relâmpagos que se formavam ao longe começaram a ficar mais próximos e a cair perto do homem. As criaturas começaram a ficar eufóricas, gritando em uma língua estranha, pulando e jogando as armas na direção da figura sentada. Tudo o que era jogado apenas batia na barreira e caia na escuridão inofensivamente.

Tarka abriu os olhos para ver o que estava acontecendo com os monstros e viu as nuvens se juntando e ficando mais espessas acima dele, os relâmpagos se aproximando a cada novo clarão. Finalmente, todas as nuvens que formavam a tempestade se juntaram para formar uma massa negra no céu que tinha Tarka como centro. Os relâmpagos que antes estavam ao longe, agora atingiam o chão perto dele. 

Foi então que um o atingiu. As criaturas ficaram mais agitadas e eufóricas do outro lado do abismo. Mas o homem já previra isso e quando o clarão passou, lá estava ele, sem nenhum arranhão e com uma aura verde ao seu redor. A mesma aura que impedia os monstros de pularem o desfiladeiro. Uma barreira.

Depois do primeiro relâmpago atingi-lo, vários se sucederam, todos sem sucesso em feri-lo. 

Tarka estava ficando assustado a cada novo choque mesmo sem sofrer nenhum dano, pois aquilo não era natural. Na verdade, era obra de alguém que sabia o que estava fazendo, sabia da sua existência ali e sabia que precisava eliminá-lo para libertar as criaturas.

Foi então que os raios pararam. O rapaz olhou para a lua e a viu sumir, parecia uma nuvem comum a princípio, mas estava se movendo muito rápido e era formada por criaturas pequenas. As pequenas criaturas começaram a rodeá-lo, então ele pôde ver que eram borboletas negras. Eram muitas e ele se viu cercado por um enxame gigantesco.

'Não é possível!' disse o homem assustado. 'Nós aprisionamos você. Você não deveria estar aqui' gritou para o alto.

As borboletas pararam de rodeá-lo e começaram a se juntar, dando forma a algo. Uma boca, que começou a falar.

'Você realmente achava que iam conseguir me prender por muito tempo aqui?' a boca falou com tom de desdém. 'Meu retorno está próximo, lembre-se disso.' Ela falava com uma voz de mulher adulta, uma voz bonita de se ouvir, mas demonstrava um sorriso diabólico, um tom sombrio.

Após falar, a grande boca se desfez e todas as criaturas que a formavam voaram para cima do guardião, que se apressou em buscar a faca no seu cinto. Ele sabia que não eram simples borboletas, eram criaturas que de alguma forma haviam conseguido passar pela sua barreira sem ele perceber. Então sabia que uma simples faca não o ajudaria, nem mesmo criar uma nova barreira. 

O homem pegou a faca e fez um corte na mão direita, então manchou o seu escudo cravado no chão. De repente, formou-se um brilho esverdeado vindo do escudo que tomou a forma de uma tartaruga do tamanho de um cavalo. A tartaruga se formou no momento que as borboletas iam atacar Tarka, então ela criou uma barreira em que as criaturas não conseguiam passar. Elas continuaram se chocando, tentando chegar nos dois, mas sem sucesso.

'Está bem feio para você ter que me chamar não é, Tarka?!' falou a tartaruga gigante.

'Está bem feio mesmo, não ocorreu como o planejado, ela conseguiu passar pela minha \textit{kekkai}. Temos que avisá-los logo.'

'Mas ainda não estão prontos, faltam trinta anos ainda para que cheguem na força máxima.'

'Não temos esse tempo todo, Casco-duro. Avise aos pais deles.'

'Mas se eu for embora você vai morrer. Não está como antes, já fazem quase mil anos. Sua mana está acabando.'

'Vou usar o resto que me resta para a \textit{kekkai} durar mais, talvez consiga dezoito anos com isso. Vá, agora!' falando isso, Casco-duro sumiu e deixou Tarka sozinho em meio ao enxame de insetos negros.

Quando a tartaruga sumiu, Tarka apenas teve tempo de criar uma barreira para afastar temporariamente os insetos, pois eles eventualmente iriam chegar nele. Entretanto, foi tempo suficiente para que ele lançasse sua magia para o escudo no chão, criando um clarão verde que destruiu todos os insetos que o estavam atacando e afastou a tempestade. 

Quando o céu ficou sem nuvens e a lua brilhou forte, Tarka não estava mais lá, apenas um esqueleto velho. Os monstros do outro lado do abismo gritavam excitados, pois não viam mais o homem que os aprisionava lá. Então começaram a pular, mas o que encontravam era uma barreira mais forte que antes, pois não apenas o jogavam na escuridão sem fim, aniquilava os que tiveram a coragem de pular. Até mesmo os insetos que conseguiram passar antes, estavam sendo evaporados. 

Naquela noite, o guardião morreu e quatro adultos tiveram o mesmo sonho com uma tartaruga gigante.

